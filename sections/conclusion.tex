

\section{CONCLUSION}
Speech Recognition has become very important in today's world. With the advancements in technology and improvements in recognition algorithms, speech has become one of the primary source of input for many applications. Speech is the most efficient and natural way of communication. So, it is intuitive that speech recognition systems have found applications in various fields. 

Interactive Voice Response (IVR) systems are one of the prominent systems that have a huge potential for use of voice signals as input to the system. With this in mind, we presented an idea for the development of an IVR system with Automatic Speech Recognition (ASR). The initial objective of the project was to develop a system capable of recognizing voice signals in Nepali Language input to the IVR system. Throughout the course of the development phase, various limitations and obstacles were encountered which prompted us to develop the system capable of recognizing words corresponding to the digits of the Nepali Language. For this, we researched on various methods of speech recognition and used the findings of these researches to develop the system. 

The project was implemented by using algorithms like Noise Reduction, Voice Activity Detection, MFCC Feature Extraction, Hidden Markov Model and Recurrent Neural Network. The overall accuracy of the system while using HMM was around 70 percentage and while using RNN was around 80 percentage. The greater accuracy of RNN is due to the fact that, RNN do not make Markov assumptions and can account for long term dependencies when modeling natural language and due to the the greater representational power of neural networks and their ability to perform intelligent smoothing by taking into account syntactic and semantic features. Though the accuracy seems to be a bit less, the accuracy is good compared to the fact that we had such less data set available. With proper amount of data set available the project can get much higher accuracy and can be implemented.



