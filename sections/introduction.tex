\pagenumbering{arabic} % Start arabic page numbers from here


\section{INTRODUCTION}


\subsection{Background}

Speech probably is the most efficient and natural way to communicate with each other. Thus, being the best way of communication, it could also be a useful interface to communicate with machines and systems like IVR system. The Interactive Voice Response (IVR) system along with the speech recognition technology can play efficient role in providing easy and efficient customer/user service. If properly implemented it can increase the user satisfaction and offer new services. Speech Recognition has now begun to dominate the market technology and is pushing away the traditional way of using hectic interfaces such as keyboards and mouse as input source to computer system. Voice command based applications will make life easier due to the fact that people will get easy and fast access to information. Therefore the popularity of automatic speech recognition system has been greatly increased. The work of speech recognition further helps in establishing  an easy way communication between interactive response system and users/ customers i.e. as a part of post processing of the speech recognizing process we can accomplish some computational task with such a system making voice input as a trigger to do some task within the system.


\setlength{\parindent}{1cm}The IVR with AVR (Automatic Voice Recognition) allows callers to obtain information and perform transactions simply by speaking naturally. Recognition of free-form conversation is not yet a reality, and speech recognition has sometimes been over-hyped in the past. However, speech recognition technology is now proving itself commercially viable in a number of customer service applications.Early speaker-dependent dictation systems had to be trained to understand the speech of one specific user. Now, speaker-independent recognition technologies allow IVR systems to interpret the speech of many users. Today's speech-enabled IVR applications serve large numbers of unknown callers without prior training of the system. Many factors are driving the emergence of speech as the IVR user interface of choice for today and tomorrow. The first is reducing labor costs. The cost of employing live customer service agents is rising at the same time that organizations are facing increased pressure to reduce the cost of serving customers. When an automated call-processing solution must be employed, a speech-enabled IVR application increases caller acceptance because it provides the friendliest and fastest self-service alternative to speaking with a customer service agent. 

In this project our focus is to build such an application where users can simply command the IVR system with their voice and in response the system accomplishes its task as per the user request. The system could have wide range of application in various fields such as interactive response customer support center, automatic number dialer, banking assistant etc.


\subsection{Problem Statement}

Most of the works done till today on the field of IVR system has been primarily focused on the input mechanisms based on the keyboard or touch pad. In such cases it is tedious to provide the input command every time through typing of texts. This way of providing input to the computer system may be enhanced if we could provide direct speech input instead of typing. This enables in fast interaction between the system and user and therefore increases overall satisfaction of the customers. This also increases the speed of access of the information from the system.
Furthermore, English language has been widely implemented in IVR systems. This has created difficulty for people while interacting with the system. Thus by implementing the Nepali voice commands it is easier to interact and provide the input to the system.


The major focus of the project being developed is the use of direct Nepali voice command for the interactive voice response system without need of typing which then further can be applicable to real world applications like call centers, customer support systems and other several organization inquiry systems.

\subsection{Objectives}
\subsubsection{Project objectives}

The prime objective of the project being proposed is to design and build a system that a basic user can interact so that she/he can make use of voice commands to deal with system i.e. making a system that has capability of recognizing the isolated speaker words and process the request to forward the given task.

The typical objectives are listed below:
\begin{itemize}
	\item To make use of domain specific models and algorithms in field of speech recognition.
	\item To develop an interactive voice response system along with speech recognition attribute.
	\item To understand the basics of speech processing.
	\item To get knowledge on various speech recognition approaches.
	\item To get insights on speech responsive application development.
\end{itemize}

\subsubsection{Academic objectives}
Academically, the project is primarily focused on fulfilling the discipline of an engineering student as a computer engineer working on a project and gain experience as a team throughout the different phases of a project. Some Typical academic objectives of the project are :
\begin{itemize}
	\item To fulfill the requirements of the major project of B.E. in computer engineering.
	\item To design and complete a functional project that integrates various course concepts.
	\item To develop various skills related to project management like team work, resource management, documentation and time management.
	\item To get hands-on experience of working in a project as a team work.
	\item To learn about and become familiar with the professional engineering practices.
\end{itemize}
\subsection{Scope}
Interactive Voice Response (IVR) system is being more popular day by day. Such systems make it more convenient in interaction with the user and computer system and hence help in easy accomplishment of several tasks. The IVR system serves as a bridge between people and computer system. The IVR system with Automatic Voice Recognition (AVR) can make it more convenient for the users if they can command the system through their voice and this makes such system applicable in vast areas.The current system is being built for the desktop computers and won't be implemented in actual phone devices due limitation of time and research. However in future the project work can be further enhanced and has huge scope and potential for future implementations in several areas. Some of the applicable areas are discussed as follows:
\begin{itemize}
	\item Organizational inquiry desk: The system can be used in several organization for easy information access regarding the organization using the voice command.
	\item Automatic speech to text conversion: The proposed system deals only with the isolated words detection but with further enhancement in algorithms it can be used in speech to text conversion application.
	\item Speech controlled applications: we can make use of speech recognition of Nepali words in performing the task of new developed applications and hence make more user friendly applications.
	\item Speech control in embedded systems: using speech recognition technology several tasks can be controlled using voice commands in embedded systems this further increases the automation of works and hence can be very beneficial in industrial process automation.
	\item Application for Disabled people: Disabled people are another part of the population that can use speech recognition programs. It is especially useful for people who can't use their hands, from the profoundly physically impaired to people with mild repetitive stress injuries i.e. persons who might require helpers to control their environments.
\end{itemize}

